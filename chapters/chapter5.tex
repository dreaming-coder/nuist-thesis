\chapter{插入表格}
\LaTeX{} 里排版表格不如 Word 等所见即所得的工具简便和自由,不过对于不太复杂的表格来讲,完全能够胜任。

排版表格最基本的 \env{tabular} 环境用法为:
\begin{lstlisting}
  \begin{tabular}[align]{column-spec}
    <item1> & <item2> & ... \\
    \hline
    <item1> & <item2> & ... \\
  \end{tabular}
\end{lstlisting}
其中 \Arg{column-spec} 是列格式标记,在接下来的内容将仔细介绍;\texttt\& 用来分隔单元格;
\crcmd{} 用来换行;\cmd{hline} 用来在行与行之间绘制横线。


直接使用 \env{tabular} 环境的话,会\textbf{和周围的文字混排}。此时可用一个可选参数 \Arg{align} 控制垂直对齐:
\verb|t| 和 \verb|b| 分别表示按表格顶部、底部对齐,其他参数或省略不写(默认)表示居中对齐。
\begin{lstlisting}
  \begin{tabular}{|c|}
  center-\\ aligned \\
  \end{tabular},
  
  \begin{tabular}[t]{|c|}
  top-\\ aligned \\
  \end{tabular},
  
  \begin{tabular}[b]{|c|}
  bottom-\\ aligned\\
  \end{tabular} tabulars.
\end{lstlisting}
\begin{center}
	\fbox{
		\parbox{30em}{
			\centering
			\begin{tabular}{|c|}
				center-\\ aligned \\
			\end{tabular},
			\begin{tabular}[t]{|c|}
				top-\\ aligned \\
			\end{tabular},
			\begin{tabular}[b]{|c|}
				bottom-\\ aligned\\
			\end{tabular} tabulars.
		}
	}
\end{center}

但是通常情况下 \env{tabular} 环境很少与文字直接混排,而是会放在 \env{table} 浮动体环境中,并用 \cmd{caption} 命令加标题。

\section{列格式}
\env{tabular} 环境使用 \Arg{column-spec} 参数指定表格的列数以及每列的格式。
\begin{table}[htp]
	\centering
	\caption{\LaTeX{} 表格列格式}
	\begin{tabular}{*{2}{l}}
		\hline
		\textbf{列格式} & \textbf{说明} \\
		\hline
		\ttfamily l/c/r          & 单元格内容左对齐/居中/右对齐,不折行 \\
		\ttfamily p\marg{width}  & 单元格宽度固定为 \Arg{width},可自动折行 \\
		\ttfamily |              & 绘制竖线 \\
		\ttfamily @\marg{string} & 自定义内容 \Arg{string} \\
		\hline
	\end{tabular}
\end{table}

\begin{lstlisting}
  \begin{tabular}{lcr|p{6em}}
    \hline
    left & center & right
    & par box with fixed width\\
    L & C & R & P \\
    \hline
  \end{tabular}
\end{lstlisting}
\begin{center}
	\fbox{
		\parbox{20em}{
			\centering
			\begin{tabular}{lcr|p{6em}}
				\hline
				left & center & right
				& par box with fixed width\\
				L    & C      & R     & P \\
				\hline
			\end{tabular}
		}
	}
\end{center}

表格中每行的单元格数目不能多于列格式里 \texttt{l/c/r/p} 的总数(可以少于这个总数),否则出错。

\texttt{@} 格式可在单元格前后插入任意的文本,但同时它也消除了单元格前后额外添加的间距。
\texttt{@} 格式可以适当使用以充当“竖线”。特别地,\texttt{@}\marg*{} 可直接用来消除单元格前后的间距:
\clearpage
\begin{lstlisting}
  \begin{tabular}{@{} r@{:}lr @{}}
    \hline
    1  & 1 & one \\
    11 & 3 & eleven \\
    \hline
  \end{tabular}
\end{lstlisting}
\begin{center}
	\fbox{
		\parbox{20em}{
			\centering
			\begin{tabular}{@{} r@{:}lr @{}}
				\hline
				1  & 1 & one \\
				11 & 3 & eleven \\
				\hline
			\end{tabular}
		}
	}
\end{center}
另外 \LaTeX{} 还提供了简便的将格式参数重复的写法 \texttt*\marg{n}\marg{column-spec},比如以下两种写法是等效的:
\begin{verbatim}
\begin{tabular}{|c|c|c|c|c|p{4em}|p{4em}|}
\begin{tabular}{|*{5}{c|}*{2}{p{4em}|}}
\end{verbatim}
有时需要为整列修饰格式,比如整列改变为粗体,如果每个单元格都加上 \cmd{bfseries} 命令会比较麻烦。
\pkg{array} 宏包提供了辅助格式 \texttt> 和 \texttt<,用于给列格式前后加上修饰命令:

\begin{lstlisting}
  % \usepackage{array}
  \begin{tabular}{>{\itshape}r<{*}l}
    \hline
    italic & normal \\
    column & column \\
    \hline
  \end{tabular}
\end{lstlisting}
\begin{center}
	\fbox{
		\parbox{20em}{
			\centering
			\begin{tabular}{>{\itshape}r<{*}l}
				\hline
				italic & normal \\
				column & column \\
				\hline
			\end{tabular}
			
		}
	}
\end{center}

辅助格式甚至支持插入 \cmd{centering} 等命令改变 \texttt{p} 列格式的对齐方式,
一般还要加额外的命令 \cmd{array\-back\-slash} 以免出错。


\begin{lstlisting}
  % \usepackage{array}
  \begin{tabular}{>{\centering\arraybackslash}p{9em}}
    \hline
    Some center-aligned long text. \\
    \hline
  \end{tabular}
\end{lstlisting}

\begin{center}
	\fbox{
		\parbox{20em}{
			\centering
			\begin{tabular}{>{\centering\arraybackslash}p{9em}}
				\hline
				Some center-aligned long text. \\
				\hline
			\end{tabular}
		}
	}
\end{center}

\pkg{array} 宏包还提供了类似 \texttt{p} 格式的 \texttt{m} 格式和 \texttt{b} 格式,
三者分别在垂直方向上靠顶端对齐、居中以及底端对齐。
\begin{lstlisting}
  % \usepackage{array}
  \newcommand\txt{a b c d e f g h i}
  \begin{tabular}{cp{2em}m{2em}b{2em}}
    \hline
    pos & \txt & \txt & \txt \\
    \hline
  \end{tabular}
\end{lstlisting}
\begin{center}
	\fbox{
		\parbox{20em}{
			\centering
			\newcommand\txt{a b c d e f g h i}
			\begin{tabular}{cp{2em}m{2em}b{2em}}
				\hline
				pos & \txt & \txt & \txt \\
				\hline
			\end{tabular}
		}
	}
\end{center}
\clearpage
\section{列宽}
在控制列宽方面,\LaTeX{} 表格有着明显的不足:\texttt{l/c/r} 格式的列宽是由文字内容的自然宽度决定的,
而 \texttt{p} 格式给定了列宽却不好控制对齐(可用 \pkg{array} 宏包的辅助格式),
更何况列与列之间通常还有间距,所以直接生成给定总宽度的表格并不容易。

\pkg{tabularx} 宏包为我们提供了方便的解决方案。它引入了一个 \texttt{X} 列格式,类似 \texttt{p} 列格式,
不过会根据表格宽度自动计算列宽,多个 \texttt{X} 列格式平均分配列宽。
\texttt{X} 列格式也可以用 \pkg{array} 里的辅助格式修饰对齐方式:
\begin{lstlisting}
  % \usepackage{array,tabularx}
  \begin{tabularx}{14em}{|*{4}{>{\centering\arraybackslash}X|}}
    \hline
    A & B & C & D \\ \hline
    a & b & c & d \\ \hline
  \end{tabularx}
\end{lstlisting}
\begin{center}
	\fbox{
		\parbox{20em}{
			\centering
			\begin{tabularx}{14em}%
				{|*{4}{>{\centering\arraybackslash}X|}}
				\hline
				A & B & C & D \\ \hline
				a & b & c & d \\ \hline
			\end{tabularx}
		}
	}
\end{center}

\section{横线}

我们已经在之前的例子见过许多次绘制表格线的 \cmd{hline} 命令。另外 \cmd{cline}\marg*{\Arg{i}-\Arg{j}} 用来绘制跨越部分单元格的横线:
\begin{lstlisting}
  \begin{tabular}{|c|c|c|}
    \hline
    4 & 9 & 2 \\ \cline{2-3}
    3 & 5 & 7 \\ \cline{1-1}
    8 & 1 & 6 \\ \hline
  \end{tabular}
\end{lstlisting}

\begin{center}
	\fbox{
		\parbox{20em}{
			\centering
			\begin{tabular}{|c|c|c|}
				\hline
				4 & 9 & 2 \\ \cline{2-3}
				3 & 5 & 7 \\ \cline{1-1}
				8 & 1 & 6 \\ \hline
			\end{tabular}
		}
	}
\end{center}

在科技论文排版中广泛应用的表格形式是三线表,形式干净简明。
三线表由 \pkg{booktabs} 宏包支持,它提供了 \cmd{toprule}、\cmd{midrule} 和 \cmd{bottomrule} 命令用以排版三线表的三条线,以及和 \cmd{cline} 对应的 \cmd{cmidrule}。除此之外,最好不要用其它横线以及竖线:
\begin{lstlisting}
  % \usepackage{booktabs}
  \begin{tabular}{cccc}
    \toprule
    & \multicolumn{3}{c}{Numbers} \\
    \cmidrule{2-4}
    & 1 & 2 & 3 \\
    \midrule
    Alphabet & A & B & C \\
    Roman    & I & II& III \\
    \bottomrule
  \end{tabular}
\end{lstlisting}

\begin{center}
	\fbox{
		\parbox{20em}{
			\centering
			\begin{tabular}{cccc}
				\toprule
				& \multicolumn{3}{c}{Numbers} \\
				\cmidrule{2-4}
				& 1 & 2 & 3 \\
				\midrule
				Alphabet & A & B & C \\
				Roman    & I & II& III \\
				\bottomrule
			\end{tabular}
		}
	}
\end{center}
\section{合并单元格}
\LaTeX{} 是一行一行排版表格的,横向合并单元格较为容易,由 \cmd{multi\-column} 命令实现:
\begin{lstlisting}
  \multicolumn{<n>}{<column-spec>}{<item>}
\end{lstlisting}
其中 \Arg{n} 为要合并的列数,\Arg{column-spec} 为合并单元格后的列格式,只允许出现一个 \texttt{l/c/r} 或 \texttt{p} 格式。
如果合并前的单元格前后带表格线 \texttt|,合并后的列格式也要带 \texttt| 以使得表格的竖线一致。
\begin{lstlisting}
  \begin{tabular}{|c|c|c|}
    \hline
    1 & 2 & Center \\ \hline
    \multicolumn{2}{|c|}{3} & \multicolumn{1}{r|}{Right} \\ 
    \hline
    4 & \multicolumn{2}{c|}{C} \\ 
    \hline
  \end{tabular}
\end{lstlisting}
\begin{center}
	\fbox{
		\parbox{20em}{
			\centering
			\begin{tabular}{|c|c|c|}
				\hline
				1 & 2 & Center \\ \hline
				\multicolumn{2}{|c|}{3} &
				\multicolumn{1}{r|}{Right} \\ \hline
				4 & \multicolumn{2}{c|}{C} \\ \hline
			\end{tabular}
		}
	}
\end{center}

上面的例子还体现了,形如 \cmd{multicolumn}\marg*{1}\marg{column-spec}\marg{item} 的命令{\bf 可以用来修改某一个单元格的列格式。}

纵向合并单元格需要用到 \pkg{multirow} 宏包提供的 \cmd{multirow} 命令:
\begin{lstlisting}
	\multirow{<n>}{<width>}{<item>}
\end{lstlisting}
\Arg{width} 为合并后单元格的宽度,可以填 \texttt{*} 以使用自然宽度。

我们看一个结合 \cmd{cline}、\cmd{multi\-column} 和 \cmd{multi\-row} 命令的例子:

\begin{lstlisting}
  % \usepackage{multirow}
  \begin{tabular}{ccc}
    \hline
    \multirow{2}{*}{Item} & \multicolumn{2}{c}{Value} \\ \cline{2-3}
    & First & Second  \\ \hline
    A & 1 & 2 \\  \hline
  \end{tabular}
\end{lstlisting}
\begin{center}
	\fbox{
		\parbox{20em}{
			\centering
			\begin{tabular}{ccc}
				\hline
				\multirow{2}{*}{Item} & \multicolumn{2}{c}{Value} \\
				\cline{2-3}
				& First & Second 
				\\ \hline
				A & 1     & 2 \\ 
				\hline
			\end{tabular}
		}
	}
\end{center}


\section{行距控制}
\LaTeX{} 生成的表格看起来通常比较紧凑。修改参数 \cmd{array\-stretch} 可以得到行距更加宽松的表格:
\begin{lstlisting}
  \renewcommand\arraystretch{1.8}
  \begin{tabular}{|c|}
    \hline
    Really loose \\ \hline
    tabular rows.\\ \hline
  \end{tabular}
\end{lstlisting}
\begin{center}
	\fbox{
		\parbox{20em}{
			\centering
			\renewcommand\arraystretch{1.8}
			\begin{tabular}{|c|}
				\hline
				Really loose \\ \hline
				tabular rows.\\ \hline
			\end{tabular}
		}
	}
\end{center}

另一种增加间距的办法是给换行命令 \crcmd{} 添加可选参数,在这一行下面加额外的间距,适合用于在行间不加横线的表格:
\begin{lstlisting}
  \begin{tabular}{c}
    \hline
    Head lines \\[6pt]
    tabular lines \\
    tabular lines \\ \hline
  \end{tabular}
\end{lstlisting}
\begin{center}
	\fbox{
		\parbox{20em}{
			\centering
			\begin{tabular}{c}
				\hline
				Head lines \\[6pt]
				tabular lines \\
				tabular lines \\ \hline
			\end{tabular}
		}
	}
\end{center}

但是这种换行方式的存在导致了一个缺陷——{\bf{表格的首个单元格不能直接使用中括号 \text{[~]}}},
否则 \crcmd{} 往往会将下一行的中括号当作自己的可选参数,因而出错。如果要使用中括号,应当放在花括号 \marg*{} 里面。
或者也可以选择将换行命令写成 \crcmd\texttt{[0pt]}。
































