\chapter{插入图片}
\section{基本使用}
\LaTeX{} 本身不支持插图功能,需要由 \pkg{graphicx} 宏包辅助支持。
在调用了 \pkg{graphicx} 宏包以后,就可以使用 \cmd{include\-graphics} 命令加载图片了:
\begin{lstlisting}
	\includegraphics[<options>]{<filename>}
\end{lstlisting}
其中 \Arg{filename} 为图片文件名,与 \cmd{include} 命令的用法类似,文件名可能需要用相对路径或绝对路径表示。图片文件的扩展名一般可不写。另外一定要注意,\textbf{文件名里既不要有空格(类似 \cmd{include}),也不要有多余的英文点号},否则宏包在解析文件名的过程中会出错。

另外 \pkg{graphicx} 宏包还提供了 \cmd{graphics\-path} 命令,用于声明一个或多个图片文件存放的目录,
使用这些目录里的图片时可不用写路径:
\begin{lstlisting}
  % 假设主要的图片放在 figures 子目录下,标志放在 logo 子目录下
  \graphicspath{{figures/}{logo/}}
\end{lstlisting}

\cmd{includegraphics} 命令的可选参数 \Arg{options} 支持 \Arg{key}=\Arg{value} 形式赋值,常用的参数如下:
\begin{table}[htp]
	\centering
	\caption{\cmd{includegraphics} 命令的可选参数}\label{tbl:graphics-options}
	\begin{tabular}{lp{18em}}
		\hline
		\textbf{参数} & \textbf{含义} \\
		\hline
		\texttt{width=}\Arg{width}    &  将图片缩放到宽度为 \Arg{width} \\
		\texttt{height=}\Arg{height}  &  将图片缩放到高度为 \Arg{height} \\
		\texttt{scale=}\Arg{scale}    &  将图片相对于原尺寸缩放 \Arg{scale} 倍 \\
		\texttt{angle=}\Arg{angle}    &  令图片逆时针旋转 \Arg{angle} 度 \\
		\hline
	\end{tabular}
\end{table}
\clearpage
\section{排版多行多列图片}
这里给个离职,模仿写就好:
\begin{lstlisting}
  \begin{figure}[htbp]
    \centering
    \includegraphics[width=...]{...}
    \qquad
    \includegraphics[width=...]{...} \\[..pt]
    \includegraphics[width=...]{...}
    \caption{...}
    \label{...}
  \end{figure}
\end{lstlisting}
\begin{figure}[htp]
	\centering
	\fcolorbox[gray]{0}{0.96}{\parbox{10em}{\vrule width 0pt height 10ex\hfil}}
	\qquad
	\fcolorbox[gray]{0}{0.96}{\parbox{10em}{\vrule width 0pt height 12ex\hfil}}
	\par\bigskip
	\fcolorbox[gray]{0}{0.96}{\parbox{20em}{\vrule width 0pt height 12ex\hfil}}
	\caption{并排放置图片的示意。}\label{fig:parallel-fig}
\end{figure}
或者使用这种方式,个人比较喜欢下面这种:
\clearpage
\begin{lstlisting}
  \begin{figure}[t]
    \centering
    \subfloat[图1]{
      \label{fig1}
      \includegraphics[width=0.5\textwidth]{图1}
    }
    \subfloat[图2]{
      \label{fig2}
      \includegraphics[width=0.5\textwidth]{图2}
    }\\ 
    \subfloat[图3]{
      \label{fig3}
      \includegraphics[width=0.5\textwidth]{图3}
    } 
    \subfloat[图4]{
      \label{fig4}
      \includegraphics[width=0.5\textwidth]{图4}
    }
    \caption{多行多列子图}
    \label{fig}	
  \end{figure}
\end{lstlisting}
\begin{figure}[H]
	\centering
	\subfloat[图1]{
		\label{fig1}
		\fcolorbox[gray]{0}{0.96}{\parbox{10em}{\vrule width 0pt height 10ex\hfil}}
	}% 
	\subfloat[图2]{
		\label{fig2}
		\fcolorbox[gray]{0}{0.96}{\parbox{10em}{\vrule width 0pt height 10ex\hfil}}
	}\\ 
	\subfloat[图3]{
		\label{fig3}
		\fcolorbox[gray]{0}{0.96}{\parbox{10em}{\vrule width 0pt height 10ex\hfil}}
	} 
	\subfloat[图4]{
		\label{fig4}
		\fcolorbox[gray]{0}{0.96}{\parbox{10em}{\vrule width 0pt height 10ex\hfil}}
	}
	\caption{多行多列子图}
	\label{fig}	
\end{figure}